%%%%%%%%%%%%%%%%%%%%%%%%%%%%% Define Article %%%%%%%%%%%%%%%%%%%%%%%%%%%%%%%%%%
\documentclass{article}
%%%%%%%%%%%%%%%%%%%%%%%%%%%%% Using Packages %%%%%%%%%%%%%%%%%%%%%%%%%%%%%%%%%%
\usepackage[spanish]{babel}
\usepackage{geometry}
\usepackage{graphicx}
\usepackage{amssymb}
\usepackage{amsmath}
\usepackage{amsthm}
\usepackage{empheq}
\usepackage{mdframed}
\usepackage{booktabs}
\usepackage{lipsum}
\usepackage{graphicx}
\usepackage{color}
\usepackage{psfrag}
\usepackage{pgfplots}
\usepackage{bm}

%%%%%%%%%%%%%%%%%%%%%%%%%%%%%%%%%%%%%%%%%%%%%%%%%%%%%%%%%%%%%%%%%%%%%%%%%%%%%%%

% Other Settings

%%%%%%%%%%%%%%%%%%%%%%%%%% Page Setting %%%%%%%%%%%%%%%%%%%%%%%%%%%%%%%%%%%%%%%
\geometry{a4paper}

%%%%%%%%%%%%%%%%%%%%%%%%%% Define some useful colors %%%%%%%%%%%%%%%%%%%%%%%%%%
\definecolor{ocre}{RGB}{243,102,25}
\definecolor{mygray}{RGB}{243,243,244}
\definecolor{deepGreen}{RGB}{26,111,0}
\definecolor{shallowGreen}{RGB}{235,255,255}
\definecolor{deepBlue}{RGB}{61,124,222}
\definecolor{shallowBlue}{RGB}{235,249,255}
%%%%%%%%%%%%%%%%%%%%%%%%%%%%%%%%%%%%%%%%%%%%%%%%%%%%%%%%%%%%%%%%%%%%%%%%%%%%%%%

%%%%%%%%%%%%%%%%%%%%%%%%%% Define an orangebox command %%%%%%%%%%%%%%%%%%%%%%%%
\newcommand\orangebox[1]{\fcolorbox{ocre}{mygray}{\hspace{1em}#1\hspace{1em}}}
%%%%%%%%%%%%%%%%%%%%%%%%%%%%%%%%%%%%%%%%%%%%%%%%%%%%%%%%%%%%%%%%%%%%%%%%%%%%%%%

%%%%%%%%%%%%%%%%%%%%%%%%%%%% English Environments %%%%%%%%%%%%%%%%%%%%%%%%%%%%%
\newtheoremstyle{mytheoremstyle}{3pt}{3pt}{\normalfont}{0cm}{\rmfamily\bfseries}{}{1em}{{\color{black}\thmname{#1}~\thmnumber{#2}}\thmnote{\,--\,#3}}
\newtheoremstyle{myproblemstyle}{3pt}{3pt}{\normalfont}{0cm}{\rmfamily\bfseries}{}{1em}{{\color{black}\thmname{#1}~\thmnumber{#2}}\thmnote{\,--\,#3}}
\theoremstyle{mytheoremstyle}
\newmdtheoremenv[linewidth=1pt,backgroundcolor=shallowGreen,linecolor=deepGreen,leftmargin=0pt,innerleftmargin=20pt,innerrightmargin=20pt,]{theorem}{Theorem}[section]
\theoremstyle{mytheoremstyle}
\newmdtheoremenv[linewidth=1pt,backgroundcolor=shallowBlue,linecolor=deepBlue,leftmargin=0pt,innerleftmargin=20pt,innerrightmargin=20pt,]{definition}{Definition}[section]
\theoremstyle{myproblemstyle}
\newmdtheoremenv[linecolor=black,leftmargin=0pt,innerleftmargin=10pt,innerrightmargin=10pt,]{problem}{Problem}[section]
%%%%%%%%%%%%%%%%%%%%%%%%%%%%%%%%%%%%%%%%%%%%%%%%%%%%%%%%%%%%%%%%%%%%%%%%%%%%%%%

%%%%%%%%%%%%%%%%%%%%%%%%%%%%%%% Plotting Settings %%%%%%%%%%%%%%%%%%%%%%%%%%%%%
\usepgfplotslibrary{colorbrewer}
\pgfplotsset{width=8cm,compat=1.9}
%%%%%%%%%%%%%%%%%%%%%%%%%%%%%%%%%%%%%%%%%%%%%%%%%%%%%%%%%%%%%%%%%%%%%%%%%%%%%%%

%%%%%%%%%%%%%%%%%%%%%%%%%%%%%%% Title & Author %%%%%%%%%%%%%%%%%%%%%%%%%%%%%%%%
\title{\textbf{Apuntes de Calculo}}
\author{Matias Leiva Zapata}
%%%%%%%%%%%%%%%%%%%%%%%%%%%%%%%%%%%%%%%%%%%%%%%%%%%%%%%%%%%%%%%%%%%%%%%%%%%%%%%


\begin{document}
\maketitle

\section{Introduccion}
el Calculo es una rama de las matematicas que estudia las variaciones de las cantidades y sus relaciones con otras,
de esta curiosa rama de las matematicas se derivan muchas otras ramas como la Geometria, la Fisica, la Estadistica,
en este recopilatorio de apuntes se intentara abarcar distintos problemas de estas diversas areas.

\section{Problemas:}

% problema numero 2.1
%%%%%%%%%%%%%%%%%%%%%%%%%%%%%%%%%%%%%%%%%%%%%%%%%%%%%%%%%%%%%%%%%%%%%%%%%%%%%%%%%%%%%
\subsection{Calculo 1.}
\smallskip
\begin{problem}[Circulos al infinito.]

Suponga que Circulos de igual diametro estan acomodados apretadamente en n filas dentro de un triangulo equilatero.
Si A es el area del triangulo Y $A_n$ es el area total ocupada por las n filas de circulos, demuestre que:

\begin{center}

    $ \displaystyle\lim_{n \to \infty}   \frac{A_n}{A} = \frac{\pi}{2\sqrt{3}}$

    \includegraphics[width=0.23\textwidth,height = 0.2\textwidth]{Triangulo_problema2.2.png}

\end{center}

\end{problem}
%%%%%%%%%%%%%%%%%%%%%%%%%%%%%%%%%%%%%%%%%%%%%%%%%%%%%%%%%%%%%%%%%%%%%%%%%%%%%%%%
% problema numero 2.2

\begin{problem}[Pesadilla del geometra.]

Deduzca la siguiente formula de John Machin:

\begin{center}

    $\displaystyle\ 4\arctan{\frac{1}{5}} - \arctan{\frac{1}{239}} = \frac{\pi}{4}  $

\end{center}


\end{problem}
%%%%%%%%%%%%%%%%%%%%%%%%%%%%%%%%%%%%%%%%%%%%%%%%%%%%%%%%%%%%%%%%%%%%%%%%%%%%%%%%

\newpage


\subsection{Calculo 2.}
\smallskip
% problema numero 2.3
%%%%%%%%%%%%%%%%%%%%%%%%%%%%%%%%%%%%%%%%%%%%%%%%%%%%%%%%%%%%%%%%%%%%%%%%%%%%%%%%%%%%%
\begin{problem}[Demuestre que:]


\[\int_{0}^{1}(1-x^2)^n dx = \frac{2^{n}{n!}}{(2n+1)!!}\]

\end{problem}

\begin{center}

    \fbox{\parbox[a]{0.985\linewidth}{\textbf{Solucion:}
    %Para poder solucionar esta integral, se debe realizar una sustitucion de variables, 
    %para esto se debe realizar la siguiente sustitucion: u = x^2, de esta forma se obtiene la siguiente integral:


    Para poder solucionar esta integral, se debe realizar una sustitucion de variables
    para esto se debe realizar la siguiente sustitucion: $ u = x^2 $,
    de esta forma se obtiene la siguiente integral:



    \[\int_{0}^{1}{(1-x^2)}^n dx = \frac{1}{2} \int_{0}^{1}{ (1-u)}^n u^{-1/2}du \]


    Luego de esto podemos integrar por partes usando las siguientes variables,
    $ u = (1-x)^n $ y $ v = x^{1/2} $,
    de esta forma se obtiene la siguiente integral:


    \[\frac{1}{2} \int_{0}^{1}{(1-x)}^n x^{-1/2}du =  n\int_{0}^{1}{(1-x)}^{  n-1} x^{1/2}du \]


    inmediatamente podemos volver a integrar por partes usando variables similares,
    $u = n{(1-x)}^{n-1} $ y $ v = \frac{2}{3}x^{3/2} $,
    de esta forma se obtiene la siguiente integral:


    \[ n\int_{0}^{1}{(1-x)}^{  n-1} x^{1/2}dx = \frac{2}{3} n(n-1) \int_{0}^{1}{(1-x)}^{n-1}x^{3/2}dx \]


    seguidamente notamos que la integral sigue una clase de recurrencia, la cual podemos escribir como:


    % aaaa

    \[I(n) =  2n I(n-1) \]


    iterando nuevamente usando integracion por partes, nos damos cuenta que se obtiene la siguiente integral:


    \[ I(n) = \frac{4}{3}n(n-1)I(n-2) \]



    \[I(n) = \frac{4}{3}\frac{2}{5}n(n-1)(n-2)I(n-3) \]


    \[I(n) = \frac{4}{3}\frac{2}{5}\dots \frac{1}{2n+1} n(n-1)(n-2)\dots (1) I(1) \]



    \[ I(n) =\frac{2^{n}n!} {(2n+1)!!} \]

    }}
\end{center}
%%%%%%%%%%%%%%%%%%%%%%%%%%%%%%%%%%%%%%%%%%%%%%%%%%%%%%%%%%%%%%%%%%%%%5%%%%%%%%%%%%%%%
\newpage

\begin{center}
    \bigskip
    \medskip
    \fbox{\parbox[a]{0.985\linewidth}{\textbf{Solucion 2:}

            Para poder solucionar esta integral, se integra por partes, haciendo la siguiente sustitucion: $ u = (1-x^2)^n $, y $ v = x $, de esta forma se obtiene la siguiente integral:


            \[\int_{0}^{1}{(1-x^2)}^n dx = x(1-x^2)^{n-1}\Big|_0^1 - n(-2)\int_{0}^{1}{x^2(1-x^2)}^{n-1} dx\]


            \[\int_{0}^{1}{(1-x^2)^n dx} =  2n\int_{0}^{1}x^2(1-x^2)^{n-1}dx\]


            \[\mathcal{I}(n) =  2n\int_{0}^{1}(x^2 -1 + 1){(1-x^2)}^{n-1}dx \]

            \[\mathcal{I}(n) = 2n\mathcal{I}(n) - 2n\mathcal{I}(n-1)\]

            \[\mathcal{I}(n) = \frac{2n}{2n+1}\mathcal{I}(n-1)\]

            \[\mathcal{I}(n) = \frac{2n}{2n+1}\frac{2(n-1)}{2(n-1)+1}\mathcal{I}(n-2)\]

            \[\mathcal{I}(n) = \frac{2^n n!}{(2n+1)!!}\]

        }}
    %%%%%%%%%%%%%%%%%%%%%%%%%%%%%%%%%%%%%%%%%%%%%%%%%%%%%%%%%%%%%%%%%%%%%%%%%%%%%%%%%%%%%
\end{center}

%problema 2.4
%%%%%%%%%%%%%%%%%%%%%%%%%%%%%%%%%%%%%%%%%%%%%%%%%%%%%%%%%%%%%%%%%%%%%%%%%%%%%%%%%%5%%
\newpage
\begin{problem}[Calcule la siguiente integral.]

\begin{center}

    \[\displaystyle\ \int_{0}^{\infty }\frac{x^2 + 1 }{x^4 + 1}dx\]

\end{center}

\end{problem}
%%%%%%%%%%%%%%%%%%%%%%%%%%%%%%%%%%%%%%%%%%%%%%%%%%%%%%%%%%%%%%%%%%%%%%%%%%%%%%%%%%%%%
\begin{center}

    \fbox{\parbox[a]{0.985\linewidth}{\textbf{Solucion:}



    \[ \displaystyle \int_{0}^{\infty }\frac{x^2 + 1 }{x^4 + 1}dx = \int_{0}^{\infty }\frac{1 + \frac{1}{x^2} }{x^2 + \frac{1}{x^2}}dx\]

    \[\displaystyle \int_{0}^{\infty }\frac{1 + \frac{1}{x^2} }{x^2 + \frac{1}{x^2}}dx = \int_{0}^{\infty }\frac{1 + \frac{1}{x^2} }{(x + \frac{1}{x})^2 + 2 }dx\]

    Para poder seguir resolviendo esta integral, usaremos una sustitucion $ u = x + \frac{1}{x} $, y de esta forma se obtiene la siguiente integral:

    \[\displaystyle \int_{0}^{\infty }\frac{1 + \frac{1}{x^2} }{(x + \frac{1}{x})^2 + 2 }dx = \int_{-\infty}^{\infty }\frac{1}{u^2 + 2 }du\]

    \[\int_{-\infty}^{\infty }\frac{1}{u^2 + 2 }du = 2 \int_{0}^{\infty }\frac{1}{u^2 + 2 }du\]

    Para poder terminar la integral, usamos una sustitucion $\sqrt{2}a = u$, y de esta manera, obtenemos la integral:

    \[\int_{0}^{\infty }\frac{1}{u^2 + 2 }du = 2\sqrt{2} \int_{0}^{\infty }\frac{1}{2a^2 + 2 }da\]

    \[\sqrt{2} \int_{20}^{\infty }\frac{1}{a^2 + 1 }da = \sqrt{2} \arctan(x)\Big|_0^{\infty} = \frac{\pi}{\sqrt{2}}\]

    }}

\end{center}
%%%%%%%%%%%%%%%%%%%%%%%%%%%%%%%%%%%%%%%%%%%%%%%%%%%%%%%%%%%%%%%%%%%%%%%%%%%%%%%%%%%%%
% Problema 2.5
\newpage
%%%%%%%%%%%%%%%%%%%%%%%%%%%%%%%%%%%%%%%%%%%%%%%%%%%%%%%%%%%%%%%%%%%%%%%%%%%%%%%%%%%%%
\begin{problem}[Sumatoria.]

Determine el intervalo de convergencia de la siguiente sumatoria y calcule la suma en terminos de $x$.

\begin{center}

    $\displaystyle\sum_{n=0}^{\infty}n^{3}x^n $

\end{center}


\end{problem}
%%%%%%%%%%%%%%%%%%%%%%%%%%%%%%%%%%%%%%%%%%%%%%%%%%%%%%%%%%%%%%%%%%%%%%%%%%%%%%%%%%%%%
\begin{center}
    \fbox{\parbox[a]{0.985\linewidth}{\textbf{Solucion:}

    Para poder determinar el intervalo de convergencia de la sumatoria, se puede utilizar el test de la razon.


    \[ \lim_{n \to \infty} \frac{{(n+1)}^{3}x^{n+1}}{n^{3}x^n} = \lim_{n \to \infty} x\frac{{(n+1)}^{3}}{n^{3}}  \]

    \[  \lim_{n \to \infty} x(\frac{n+1}{n})^3  =  x(\lim_{n \to \infty} \frac{n+1}{n})^3  =  x(1)\]

    \begin{center}
        Al obtener nuestro resultado, obtenemos que nuestra serie converge con:
    \end{center}  \[\left\lvert x\right\rvert < 1 \]

    Para obtener el valor de la suma, nos basamos del hecho de una suma geometrica, sabemos que, la formula de una suma geometrica es igual a:

    \[\frac{1}{1-x} =  \sum_{n=0}^{\infty}x^n\]

    \[\frac{d}{dx}\frac{1}{1-x} =  \frac{d}{dx}\sum_{n=0}^{\infty}x^n\]

    \[\frac{-1}{{(1-x)}^2} =  \sum_{n=0}^{\infty}nx^{n-1}\]

    Luego multiplicando la ecuacion por x en ambos lados, se tiene que:

    \[\frac{-x}{{(1-x)}^2} =  \sum_{n=0}^{\infty}nx^{n}\]

    luego volvemos a derivar, para obtener que:

    \[\frac{1+x}{{(1-x)}^3} =  \sum_{n=0}^{\infty}n^{2}x^{n-1}\]

    \[\frac{(1+x)x}{{(1-x)}^3} =  \sum_{n=0}^{\infty}n^{2}x^{n}\]

    \[\frac{d}{dx}\frac{(1+x)x}{{(1-x)}^3} = \frac{d}{dx} \sum_{n=0}^{\infty}n^{2}x^{n}\]

    \[\frac{x^2 + 4x +1 }{{(1-x)}^4} = \sum_{n=0}^{\infty}n^{3}x^{n-1}\]

    \[\frac{(x^2 + 4x +1)x}{{(1-x)}^4} = \sum_{n=0}^{\infty}n^{3}x^{n}\]
    }}



    %%%%%%%%%%%%%%%%%%%%%%%%%%%%%%%%%%%%%%%%%%%%%%%%%%%%%%%%%%%%%%%%%%%%%%%%%%%%%%%%%%%%%
    % %%%%%%%% Problema 2.6
    %%%%%%%%%%%%%%%%%%%%%%%%%%%%%%%%%%%%%%%%%%%%%%%%%%%%%%%%%%%%%%%%%%%%%%%%%%%%%%%%%%%%%


    \newpage
    \begin{problem}[Calcule la siguiente integral. MIT integration bee finals 2023.]

    \begin{center}

        \[\displaystyle\ \int_{0}^{\frac{\pi}{2}}\frac{\sqrt[3]{\tan{x}}}{{(\sin{x}+ \cos{x})}^2}dx\]

    \end{center}

    \end{problem}
    \fbox{\parbox[a]{0.985\linewidth}{\textbf{Solucion:}
        Para resolver esta integral, vamos a necesitar unas manipulaciones algebraicas,
         para esto, primero transformaremos el denominador en terminos de una tangente
         \[\displaystyle\ \int_{0}^{\frac{\pi}{2}}\frac{\sqrt[3]{\tan{x}}}{{(\sin{x}+ \cos{x})}^2}dx = \displaystyle\ \int_{0}^{\frac{\pi}{2}}\frac{\sqrt[3]{\tan{x}}}{{\cos(x)}^2{(\tan{x}+ 1)}^2}dx\]
        \[\displaystyle\ \int_{0}^{\frac{\pi}{2}}\frac{\sqrt[3]{\tan{x}}}{{(\tan{x}+ 1)}^2}{\sec(x)}^{2}dx\]
    }}




\end{center}

%%%%%%%%%%%%%%%%%%%%%%%%%%%%%%%%%%%%%%%%%%%%%%%%%%%%%%%%%%%%%%%%%%%%%%%%%%%%%%%%%%%%%
% %%%%%%%% Problema 2.7
%%%%%%%%%%%%%%%%%%%%%%%%%%%%%%%%%%%%%%%%%%%%%%%%%%%%%%%%%%%%%%%%%%%%%%%%%%%%%%%%%%%%%

\newpage
\begin{problem}[Demuestre que la integral tiene las siguientes cotas:]

\begin{center}

    \[\displaystyle\    \frac{1}{7\sqrt{2}} \leq \int_{0}^{1}\frac{{x}^6}{\sqrtsign{1+{x}^2}}dx \leqslant \frac{1}{7}\]

\end{center}

\end{problem}
\begin{center}


    \fbox{\parbox[a]{0.985\linewidth}{\textbf{Solucion:}

    Tenemos que para el intervalo \[x\in{[0,1]}\]
    \[1 \leqslant \sqrt{1+x^2} \; \; \; \forall x \in{[0,1]} \]

    \[ \frac{1}{\sqrt{1+x^2}} \leqslant 1  \; \; \; \forall x \in{[0,1]} \]

    \[ \frac{{x}^6}{\sqrt{1+x^2}} \leqslant x^6  \; \; \; \forall x \in{[0,1]} \]

    \[ \int_{0}^{1} \frac{{x}^6}{\sqrt{1+x^2}} \,dx  \leqslant   \int_{0}^{1} x^6  \,dx    \]

    \[ \int_{0}^{1} \frac{{x}^6}{\sqrt{1+x^2}} \,dx  \leqslant  \frac{x^7}{7} \Big|_0^{1} = \frac{1}{7}  \]

    \medskip
    Para la desigualdad en el otro lado, tenemos que:

    \[x^6 \leqslant 1  \; \; \; \forall x \in{[0,1]}  \]

    \[x^6 \leqslant x^6\sqrt{1+x^2} \leqslant x^6\sqrt{2} \; \; \; \forall \in{[0,1]}  \]

    \[\frac{x^6 }{\sqrt{2}} \leqslant \frac{x^6\sqrt{1+x^2}}{\sqrt{2}} \leqslant x^6  \; \; \; \forall \in{[0,1]}  \]

    \[\frac{x^6 }{\sqrt{2(1+x^2)}} \leqslant \frac{x^6}{\sqrt{2}} \leqslant \frac{x^6}{\sqrt{1+x^2}} \; \; \; \forall \in{[0,1]}  \]

    \[\int_{0}^{1} \frac{x^6}{\sqrt{2}}  \,dx \leqslant \int_{0}^{1} \frac{x^6}{\sqrt{1+x^2}}  \,dx  \; \; \; \forall \in{[0,1]}  \]

    \[\frac{x^7}{7\sqrt{2}} \Big|_0^{1} = \frac{1}{7\sqrt{2}} \leqslant \int_{0}^{1} \frac{x^6}{\sqrt{1+x^2}}  \,dx  \; \; \; \forall \in{[0,1]}  \]

    Finalmente:
    \[\frac{1}{7\sqrt{2}} \leqslant \int_{0}^{1} \frac{x^6}{\sqrt{1+x^2}}  \,dx \leqslant \frac{1}{7}   \; \; \; \forall \in{[0,1]}  \]

    }}



    %%%%%%%%%%%%%%%%%%%%%%%%%%%%%%%%%%%%%%%%%%%%%%%%%%%%%%%%%%%%%%%%%%%%%%%%%%%%%%%%%%%%%
    % %%%%%%%% Problema 2.8
    %%%%%%%%%%%%%%%%%%%%%%%%%%%%%%%%%%%%%%%%%%%%%%%%%%%%%%%%%%%%%%%%%%%%%%%%%%%%%%%%%%%%%

    \newpage
    \begin{problem}
    Para cualquier funcion integrable $f(x)$, muestre que:
    \[ \lim_{n \to \infty} \frac{1}{n} \int_{0}^{n} f(\frac{\lfloor x\rfloor }{n})  \,dx  = \int_{0}^{1}f(x)  \,dx \]

    \end{problem}

    \fbox{\parbox[a]{0.985\linewidth}{\textbf{Solucion:}
    Para la funcion piso, se tiene que para todo x,
    \[k \leqslant \left\lfloor x\right\rfloor < k+1\]

    \[\lim_{n \to \infty} \frac{1}{n} \int_{0}^{n} f(\frac{\lfloor x\rfloor }{n})  \,dx  =
        \lim_{n \to \infty} \frac{1}{n} \int_{0}^{1} f(\frac{\lfloor x\rfloor }{n})  \,dx + \frac{1}{n} \int_{1}^{2} f(\frac{\lfloor x\rfloor }{n})  \,dx  \ldots
        + \frac{1}{n} \int_{n-1}^{n} f(\frac{\lfloor x\rfloor }{n})  \,dx \]


    \[\lim_{n \to \infty} \frac{1}{n} \int_{0}^{n} f(\frac{\lfloor x\rfloor }{n})  \,dx  =
        \lim_{n \to \infty}\frac{1}{n}\sum_{k= 0}^{n}\-\int_{k}^{k+1} f(\frac{\lfloor x\rfloor }{n})  \,dx   \]



    \[\lim_{n \to \infty}\frac{1}{n}\sum_{k= 0}^{n}\-\int_{k}^{k+1} f(\frac{\lfloor x\rfloor }{n})  \,dx   =  \lim_{n \to \infty}\frac{1}{n}\sum_{k= 0}^{n}\-\int_{k}^{k+1} f(\frac{k}{n})  \,dx  \]

    \[ \lim_{n \to \infty}\frac{1}{n}\sum_{k= 0}^{n} f(\frac{k}{n})  \int_{k}^{k+1}\,dx =  \lim_{n \to \infty}\frac{1}{n}\sum_{k= 0}^{n} f(\frac{k}{n}) \-\-\- x\Big|_k^{k+1} \]

    \[ \lim_{n \to \infty}\sum_{k= 0}^{n} \frac{1}{n}f(\frac{k}{n}) \]
    Luego al llegar a esta ecuacion, nos podemos dar cuenta que esta es la suma de riemann, de la integral $f(x)$, por lo tanto,
    \[ \lim_{n \to \infty} \frac{1}{n} \int_{0}^{n} f(\frac{\lfloor x\rfloor }{n})  \,dx  = \int_{0}^{1}f(x)  \,dx \]


    }}

    %%%%%%%%%%%%%%%%%%%%%%%%%%%%%%%%%%%%%%%%%%%%%%%%%%%%%%%%%%%%%%%%%%%%%%%%%%%%%%%%%%%%%
    % %%%%%%%% Problema 2.9     ¡¡¡¡¡¡¡¡¡¡¡¡¡¡¡¡¡¡¡¡¡¡¡¡¡¡


    


    
    %%%%%%%%%%%%%%%%%%%%%%%%%%%%%%%%%%%%%%%%%%%%%%%%%%%%%%%%%%%%%%%%%%%%%%%%%%%%%%%%%%%%%


\end{center}

%%%%%%%%%%%%%%%%%%%%%%%%%%%%%%%%%%%%%%%%%%%%%%%%%%%%%%%%%%%%%%%%%%%%%%%%%%%%%%%%%%%%%



\end{document}